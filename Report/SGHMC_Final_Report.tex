\documentclass[12pt]{article}
\usepackage[letterpaper, scale=0.85]{geometry} % Reduce document margins
\usepackage{graphicx}
\usepackage{amsmath}
\usepackage{amsfonts}
\usepackage{amssymb}
\usepackage{bm}
\usepackage{algorithm2e}
\usepackage{float}

%opening
\setlength{\parindent}{0pt}

% Custome latex commands
\newcommand{\prob}[1]{\mathrm{Pr}\left(#1\right)}
\newcommand{\braces}[1]{\left\lbrace#1 \right\rbrace}

\begin{document}
\vspace{-1in}
\author{\bf Sta663 Final Project}
\title{\bf Stochastic Gradient Descent Hamiltonian Monte Carlo Applied to Bayesian Logistic Regression}
\date{Gilad Amitai and Beau Coker}
%\date{\today}
\maketitle 
%\thispagestyle{empty}


\begin{abstract}
	Hamiltonian Monte Carlo (HMC) is a Markov chain Monte Carlo algorithm for drawing samples from a probability distribution where proposed values are computed using Hamiltonian dynamics to find values of high acceptance probabilities. They allow us to explore sample states more efficiently than random walk proposals, but are limited by the expensive computation of the gradient of the potential energy function. Chen, Fox, and Guestrin propose the method Stochastic Gradient Hamiltonian Monte Carlo (SGHMC), a HMC algorithm that uses a subset of the data to compute the gradient. The authors find that the stochastic gradient is noisy and correct this with a friction term.

	In this project, we adapt the SGHMC to be used for Bayesian Logistic regression, implement this method in Python, optimize the code for computational efficiency, validate our approach using simulated data, and apply the algorithm to real world classification problems.
	
	Keywords: Hamiltonian Monte Carlo, Stochastic Gradient Hamiltonian Monte Carlo, Pima Indians Diabetes Dataset, Hockey Puck, Logistic Regression, Markov chain Monte Carlo
\end{abstract}

\section{Background}
Because this is a project about Hamiltonian Monte Carlo, imagine a frictionless puck on an icy surface of varying heights. The state of this puck is given by it's momentum $\bm{q}$ and position $\bm{p}$. The potential energy of the puck $U$ will be a function of only its height, while the kinetic energy will be a function of its momentum $K(q)=\frac{|\bm{q}|^2}{2m}$. If the ice is flat, the puck will move with a constant velocity. If the ice slopes upwards, the kinetic energy will decrease as the potential energy increases until it reaches zero, at which point it will slide back down. In the context of Bayesian statistics, we can think of the position as the posterior distribution we want to sample from, and the momentum variable are artificial constructs that allow us to efficiently move around our space. The Hamiltonian $H(q, p)$ will be the sum of the potential and kinetic energy, and its partial derivatives will determine how $q$ and $p$ change overt ime $t$
\begin{equation*}
	\begin{split}
	\frac{d p_i}{d t}=\frac{\partial H}{\partial q_i}\\
	\
	\end{split}
\end{equation*}

\section{Description of Algorithm}
In their \textit{Stochastic Gradient Hamiltonian Monte Carlo}, Chen, Fox, and Guestrin propose using a subset $\tilde{\mathcal{D}}$ of the entire dataset $\mathcal{D}$ to compute
\begin{equation*}
	\Delta\tilde{U}(\theta)=-\frac{|\mathcal{D}|}{|\tilde{\mathcal{D}}|}\sum_{x\in\tilde{\mathcal{D}}}\Delta\log p(x|\theta) - \Delta\log p(\theta)
\end{equation*}
which can then be used in the Hamiltonian Monte Carlo equations in the stead of the gradient $\Delta U(\theta)$.

Logistic regression assigns the probability of success to a dichotomous response variable

\begin{equation*}
	\prob{y_i=1|\bm{x}_{i},\bm{\beta}}=\frac{\exp\braces{\bm{x}_i^T\bm{\beta}}}{1+\exp\braces{\bm{x}_i^T\bm{\beta}}}
\end{equation*}

where $\bm{x}_i$ is a vector of length $p$ covariates for data point $i$ and $\bm{\beta}$ is a vector of regression coefficients of length $p$. In a Bayesian framework, we would assign the a prior distribution on our unknown parameters $P(\bm{\beta})\sim\mathcal{N}(0, \sigma^2)$ where, for the purposes of our project, $\sigma^2$ is known. The corresponding posterior will be proportional to $P(\theta)\prod_{i=1}^n\prob{y_i|\bm{x}_i,\bm{\beta}}$, which would give us the potential energy function
\begin{equation*}
U(\bm{\beta})=-\log\left[P(\bm{\beta})\right] - \sum_{i=1}^n\log\left[\prob{y_i|\bm{x}_i,\bm{\beta}}\right]=\sum_{j=1}^p\frac{\beta_j^2}{2\sigma^2}-\sum_{i=1}^n\left[y_i(\bm{x}_i^T\bm{\beta}-\log(1+\exp\braces{\bm{x}_i^T\bm{\beta}})\right]
\end{equation*}
and gradient components
\begin{equation*}
\frac{\partial U}{\partial \beta_j}=\frac{\beta_j}{\sigma^2}-\sum_{i=1}^{n}x_{ij}\left[y_i-\frac{\exp\braces{\bm{x}_i^T\bm{\beta}}}{1+\exp\braces{\bm{x}_i^T\bm{\beta}}}\right]. 
\end{equation*}
In practice, the stochastic gradient is noisy since it is an approximation of the gradient. The paper suggests introducing a friction term to the momentum to dampen the movement of the chain. The algorithm will take a user specified friction term $C$ that is element-wise bigger than the noise model $B$. The noise model is unknown but can be set to zero for simplicity. The final algorithm will be:

\begin{algorithm}[H]
	\KwIn{Starting position $\theta^{(1)}$ and step size $\epsilon$.}
	\For{t=1, 2, \dots} {
		Sample momentum $r^{(1)}\sim\mathcal{N}(0, M)$\\
		\For{i=1,\dots,m} {
			\theta_i=\theta_{i-1}+\epsilon M^{-1}r_{i-1}\\
			r_i = r_{i-1} + \epsilon \Delta\tilde{U}(\theta_i)-\epsilon C M^{-1}r_{i-1}+\mathcal{N}(0, 2(C-\hat{B})\epsilon)\\
		}
		(\theta^{t+1}, r^{t+1})=(\theta_m, r_m)\\
	}
\end{algorithm}

\section{Optimization}
Any MCMC algorithm is inherently sequential, and so can't be parallelized (though multiple chains can be run at the same time). Each parameter's full conditional distribution depends on other parameters, so loops are a natural implementation. Unfortunately, loops are quite slow in Python. On the other hand, a compiled language has no such issues. In this section, we discuss an alternative, optimized version of our code that is written in C++ and is callable from Python via the Pybind11 package.

We feel this is a nice Both libraries are written with the same input syntax and function names, so they can be easily exchanged. A small difference is that the potential energy function and gradient function cannot be supplied by the user in the C++ implementation.\\

Note that one drawback to a C++ implementation is the limited availability of easy-to-use random sampling functions. To complete our algorithm, we wrote a random multivariate normal function based on a Cholesky decomposition (necessary for the momentum updates and noise terms) and a random sampling function (necessary for the stochastic gradient descent). Unfortunately, the random sampling function is quite slow, dampening the impact from the low-level speedup.\\

To compare the performance of our two implementations, we use the Pima Indian data again and make use of the {\tt \%timeit} magic function. Unfortunately, while the {\tt \%prun} profiler provides useful information on the Python implementation, it does not work on the C++ implementation.\\
	
Starting with the stochastic gradient function, on 1000 loops we find that the Python implementation has a best of 10 of $97.9 \mu s$ per loop while the C++ version has a best of 10 of $45.6 \mu s$ per loop. This is less than the speedup we would usually expect from a compiled language. The reason has to do with the innefficient random sampling procedure used in our C++ code. This could certainly be improved.\\

Using 10 loops and looking at the best of 3 this time, we find that the Python implementation of the sghmc function takes about $5.12 s$ per loop, while the C++ version takes about $663 ms$. In both cases, we used a batch size of 100, a number of time steps of 20. Here we can see the dramatic improvements of the lower level implementation.\\

As discussed, {\tt \%prun} will not work for the C++ version, so we can only examine the results for the unoptimized Python version. We find that function spends 0.929 seconds of its total time (out of 5.865 seconds) in the multivariate normal function. The stochastic gradient is close behind at 0.801. The logistic shows up with a total time of 0.280.\\

Further of the comparison and the {\tt \%prun} results are available in a Ipython notebook on the provided Github repository.



\section{Application to Simulated Data}
To test our implementation of the algorithm, we simulated data with 50 covariates and 500 observations, where the covariates were sampled from a normal distribution with mean zero and variances 25, 5, and 0.4 sampled form a multinomial distribution with probability vector $(.05, .05, .9)$. Data was normalized to have mean zero and a standard deviation of one. No intercept was fit.\\

In general, we find that the algorithm requires careful attention to the step rate $\eta$, the number of time steps $m$, and the size of the minibatch for the stochastic gradient. While these parameters don't need to be calibrated precisely, they need to be on the right order of magnitude or the algorithm will not converge or mix poorly.

We can see that using the Hamiltonian Monte Carlo algorithm produced suboptimal mixing:

\begin{figure}[H]
	\centering
	\begin{minipage}{0.45\textwidth}
		\centering
		\includegraphics[width=0.9\textwidth]{hmc-energy-sim.pdf} % first figure 
	\end{minipage}\hfill
	\begin{minipage}{0.45\textwidth}
		\centering
		\includegraphics[width=0.9\textwidth]{hmc-trace-sim.pdf} % second figure 
	\end{minipage}
\end{figure}

We can see that using the Stochastic Gradient Hamiltonian Monte Carlo algorithm produced good mixing:

\begin{figure}[H]
	\centering
	\begin{minipage}{0.45\textwidth}
		\centering
		\includegraphics[width=0.9\textwidth]{sghmc-energy-sim.pdf} % first figure
	\end{minipage}\hfill
	\begin{minipage}{0.45\textwidth}
		\centering
		\includegraphics[width=0.9\textwidth]{sghmc-trace-sim.pdf} % second figure
	\end{minipage}
\end{figure}

We compared the coefficients produced by each algorithm to the MLE estimates to check for accuracy:

\begin{figure}[H]
	\centering
	\includegraphics[width=0.45\textwidth]{coefs-sim.pdf}
\end{figure}

We see that all coefficients are fairly similar and that the SGHMC is closer to the GD and MLE estimates than the HMC estimates.

\section{Application to Real Data}
traceplots
coefficients
description of pima dataset

%\section{Comparative Analysis}
%\input{comparative.tex}

\section{Discussion and Conclusion}
In implementing this algorithm, we found that the algorithm was very sensitive to its tuning parameters, such as the learning rate $\epsilon$ and the mass matrix $M$. In a more complete implementation of Stochastic Gradient Hamiltonian Monte Carlo, we would want to include adaptive tuning parameters, like those that are used in STAN. Our results from the simulated and real data validated our code and model. While we used logistic regression, the package we have written is modular enough to let the user enter his or her own model.

\section{Bibliography}


\end{document}