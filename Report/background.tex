Because this is a project about Hamiltonian Monte Carlo, imagine a frictionless puck on an icy surface of varying heights. The state of this puck is given by it's momentum $\bm{q}$ and position $\bm{p}$. The potential energy of the puck $U$ will be a function of only its height, while the kinetic energy will be $K(q)=\frac{|\bm{p}|^2}{2m}$. If the ice is flat, the puck will move with a constant velocity. If the ice slopes upwards, the kinetic energy will decrease as the potential energy increases until it reaches zero, at which point it will slide back down. 